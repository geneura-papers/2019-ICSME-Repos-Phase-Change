\documentclass[conference]{IEEEtran}
\IEEEoverridecommandlockouts
\usepackage{url} 
\usepackage{cite}
\usepackage{amsmath,amssymb,amsfonts}
\usepackage{algorithmic}
\usepackage{graphicx}
\usepackage{textcomp}
\usepackage{xcolor}
\def\BibTeX{{\rm B\kern-.05em{\sc i\kern-.025em b}\kern-.08em
    T\kern-.1667em\lower.7ex\hbox{E}\kern-.125emX}}

\begin{document}

\title{Power laws in code repositories: A skeptical approach\\
	\thanks{This paper has been supported in part by
		project DeepBio (TIN2017-85727-C4-2-P)}
}

\author{\IEEEauthorblockN{1\textsuperscript{st} Bartolom\'{e} Ortiz}
	\IEEEauthorblockA{\textit{Geneura Team,ETSIIT and CITIC} \\
		\textit{University of Granada}\\
		Granada, Spain \\
		bortiz@ugr.es}
	\and
	\IEEEauthorblockN{2\textsuperscript{nd} J. J. Merelo}
	\IEEEauthorblockA{\textit{Geneura Team,ETSIIT and CITIC} \\
		\textit{University of Granada}\\
		Granada, Spain \\
		jjmerelo@ugr.es}
}

\maketitle

\begin{abstract}
  Software development using control systems is one of the main ways
  of creation in our days.  Traditionally, the study of the evolution
  of this type of systems is usually done from the perspective of
  self-organized complex systems, since the version-control tools used
  allow a development oriented to the community without a fixed
  leader.

  One of the essential characteristics needed to understand this type
  of systems' behaviour and to know if we are truly dealing with an
  auto-organized community is the appearance or not of a power law in
  the events'distributions (in this case, events are understood as
  commits, that is, additions to the code by users).

  But we have to be extremely cautious with this kind of approach.
  
  The existence or not of this type of distributions should be always
  analyzed mathematically to minimize posible sources of bias.
  Thankfully we can use the concept of hypothesis tests for that
  purpose, that is, checking what is the likelihood that our data
  follow the distribution of power law instead of some other with
  similar characteristics.

  However, the conclusions of these tests may not shed light on
  whether we are in a state of self-organization.  Software
  development is a process that evolves over time and analysis at a
  specific moment is not significant.  Our approach in this work uses
  a temporal analysis over the evolution of commit's distribution in
  software projects. For that, we have selected a set of repositories
  that exhibit singular behaviors such as a large number of
  contributions versus few or inconclusive results in statistical
  tests.

  Our study seeks to provide conclusive evidence about the state of
  the chosen repositories, answering wether they have not reached a
  state of self-organization in all its evolution (that is, never
  present power law distributions) or, if they have done so, but due
  to other code insertions and to the temporary nature of the creative
  process, this self-organized state has been altered.
  

\end{abstract}

\begin{IEEEkeywords}
	Complex systems, self-organizing systems, self-organized criticality, power laws, software development, 
	software repositories
\end{IEEEkeywords}


%%%%%%%%%%%%%%%%%%%%%%%%%%%%%%%   INTRODUCTION   %%%%%%%%%%%%%%%%%%%%%%%%%%%%%%%
\section{Introduction}\label{introduction}
. 

% --------------------------------------------------------


\section{State of the art}\label{soa}



% --------------------------------------------------------


\section{Methodology}
\label{sec:method}


\subsection{Hypothesis tests}



\subsection{Classification}

\subsection{Visualization}



% --------------------------------------------------------


\section{Results}
\label{res}


% --------------------------------------------------------


\section{Conclusions}\label{conc}


\bibliographystyle{apalike}
\bibliography{geneura,biblio}

\end{document}
